\documentclass{article}

\usepackage[backend=biber]{biblatex}

% -- Title stuff --
\title{Concept Suggestor}
\date{17th of May, 2017}
\author{Rapha\"el Claasen}

% -- Bibliography --
\addbibresource{bibliography.bib}

% -- Actual Document --
\begin{document}

\maketitle

\begin{abstract}
This is the abstract.
\end{abstract}

{\bf Keywords:} one, two, three

\section{Introduction}

ATM is a cross-disciplinary field that requires analysts from multiple domains such as security, safety, and business. When discussing problems about the ATM field, analysts often create domain-specific models that employ both shared and domain-specific concepts. To facilitate these discussions it is important that domain-specific models from one domain can be used in communication with other domains. This can be done by ensuring that domains use similar concepts in their models where possible.

It is difficult for analysts from different domains to communicate about their proposed solutions to a given ATM problem. This difficulty is partially caused by a difference in concepts used in their modeling which reduces the usefulness of these models in communication with other domains.

\section{Related Work}

SpaCy is one of the fastest and most accurate publicly available Natural Language Processing (NLP) toolkits available.\cite{choi2015depends}
SpaCy uses GloVe to create word vectors. GloVe is an unsupervised learning algorithm for obtaining vector representations for words.\cite{pennington2014glove} These word vectors can be used to model semantic similarity of words.

\section{Algorithms}

These are the algorithms.

\section{Evaluation Methodology}

\begin{itemize}
\item How to measure similarity in concepts used in the various domain-specific models?
\item How to determine which are relevant concepts from other models that should be suggested to a modeler in order to improve the alignment between the models?
\item How to determine which are relevant concepts from elsewhere that might improve alignment between the models?
\end{itemize}

\section{Experiments}

These are the experiments.

\section{Conclusions}

These are the conclusions.

\section{Acknowledgements}

I would like to thank Fabiano Dalpiaz and Ba\c sak Aydemir for sharing expertise, valuable guidance, and encouragement with me.

\printbibliography

\end{document}